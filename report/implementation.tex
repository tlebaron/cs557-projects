\section{Implementation}
\subsection{Environment Creation}
The Docker image does not need any configuration. For isntance, ASLR is kept enabled.

The program exploited is GNU Screen 4.5. We followed the usual step to install a software on a linux system from a compressed file. After uncompressing the software folder, we run {\tt ./configure}, {\tt make} and {\tt make install}.

\subsection{Exploit Construction}
The exploit has three parts.

\begin{itemize}
\item First we create the library and the exploit file. The library corresponds to the lines 11-20 of the exploit. Its job is to give the exploit file the root privileges needed. It also unlink the {\tt /etc/ld.so.preload} file so that the library is not called more than once. The exploit file simply get root privileges and execute {\tt execvp("bin/shh")}. It corresponds to the lines 25-32.
\item Then we add the library to the {\tt /etc/ld.so.preload} file. Done line 39, it can be only done if this file doesn't exist in the first place.
\item Second we run Screen again. Before running, all the libraries in the {\tt ld.so.preload} will do so. Since Screen has root privileges, then the library added will execute, giving the exploit file all the privileges needed. Once done, we simply run the exploit file with {\tt /tmp/rootshell}.
\end{itemize}

