
\documentclass[10pt, conference, letterpaper]{IEEEtran}

\usepackage{cite}
\usepackage{amsmath,amssymb,amsfonts}
\usepackage{algorithmic}
\usepackage{graphicx}
\usepackage{textcomp}
\usepackage{xcolor}

\usepackage{todonotes,graphicx,tabu,tabularx}
\usepackage{xspace}
\usepackage{bm}
\usepackage{color}
\usepackage{breakurl}
\usepackage{listings}
\lstset{
  basicstyle=\ttfamily,
  columns=fullflexible,
  frame=single,
  breaklines=true,
  postbreak=\mbox{\textcolor{red}{$\hookrightarrow$}\space},
}

\usepackage[labelformat=simple]{subcaption}
\renewcommand\thesubfigure{(\alph{subfigure})}

\usepackage{amsmath}




\def\BibTeX{{\rm B\kern-.05em{\sc i\kern-.025em b}\kern-.08em
    T\kern-.1667em\lower.7ex\hbox{E}\kern-.125emX}}
\begin{document}

%%% Functions %%%
\graphicspath{ {images/} }
\newcommand{\sysname}{ModiPick\xspace}
\newcommand{\Brittany}[1]{{\color{red}\textbf{Brittany: \textit{#1}}}}
\newcommand{\sam}[1]{{\color{blue}\textbf{Sam: \textit{#1}}}}
\newcommand{\eat}[1]{}
%%%%%%%%%%%%



\title{PeanutPower: CS557 Project 2
%\thanks{Identify applicable funding agency here. If none, delete this.}
}

% Double blind 1.
\author{\IEEEauthorblockN{Thomas Le Baron}
\IEEEauthorblockA{\textit{Computer Science} \\
\textit{Worcester Polytechnic Institute}\\
Worcester MA, USA \\
tlebaron@wpi.edu}
\and
\IEEEauthorblockN{Brittany Lewis}
\IEEEauthorblockA{\textit{Computer Science} \\
\textit{Worcester Polytechnic Institute}\\
Worcester MA, USA \\
bfgradel@wpi.edu}
}

\maketitle

\begin{abstract}
In this work we introduce a vulnerable binary "PeanutPower" which has a vulnerable buffer overflow. We also prove that this binary can be explioted to launch shell code. We do this through overwriting the global offset table using a buffer overflow, and then using return oriented program to call mprotect and make our stack executable so that we can run our shell code. 
\end{abstract}

\section{Introduction}
This project aims to demonstrate our knowledge on code-reuse attacks. After code injection attacks, defenses like the NX bits has been developed. In this case, it enforce the fact that code passed on the stack or the heap should never be executed, since it should just be used to store in memory. With stack and heap not executable, we can not execute injected shellcode. The idea of code-reuse attack is to execute code already existing, code we did not inject. The basic code-reuse attack is to execute the system() function of the libc by passing the "/bin/sh" argument on the stack (only works on 32 bits). A more interesting subclass of code-reuse attack is named return oriented programing (ROP). It uses the fact that the control flow can jump in the middle of any kind of existing function to only executed what is needed. For instance, calling system("/bin/sh") on 64 bits means first loading the "/bin/sh" address in the right register and then jump to system(). Loading a register like rdi needs only a pop rdi instruction since we assume the attacker can take control of the content of the stack. Therefore, an attacker needs only to jump to a pop instruction into rdi followed by a return address. By writing the address of system() on the stack, the "gadget" (pop instruction and return) will simply load whatever the attacker wrote at the top of the stack into rdi before returning at the system() function.
This paper explain how we used this type of attack to get privilege escalation on a code with a simple vulnerability, the format string vulnerability. This vulnerability is not used to overwrite a return address. As a result, the canaries added at compiler time do not prevent our attack in any way. Nevertheless, we had to disble the ASLR protection to keep this exploit easily understandable.

\section{Design}
access\_to\_the\_peanut\_cave is a binary which involves a series of three challenges that need to be correctly completed in order to enter the secret peanut cave (and launch /bin/shh using execve to gain privilege escalation).

\subsection {Lock 1: Random password}
Our first lock was inspired by a CTF binary at THOMAS PUT THE LOCATION WHERE YOU FOUND THAT THINGY. In access\_to\_the\_peanut\_cave we ask for a key and then perform an xor with a key that we \''randomly\'' generated using the rand function and compare it to a value 0x9000ddo9 (goood dog). 

\textbf{Rand():} The core of the vulnerability comes from our incorrect usage of the rand function. We forgot to actually seed our random function! As a result, it will be given the default seed of 0. This means that we can easily create a separate script that calls rand the same way, and gets the same ``random'' number. This will allow us to get the correct value to open this lock. 

\textbf{Scanf():} With this lock we ran into an issue later in our code. In the original challenge binary, the way that scanf was called, it left behind a newline character. This meant that our gets function did not work since it stopped immediately at the new line character left behind by scanf. We changed the call to scanf to expect a new line character so that it would not leave behind issues that would affect our program's execution later on.

\subsection{Lock 2: File IO Lock}
Our second lock was inspired by a CTF binary at THOMAS PUT THE LOCATION WHERE YOU FOUND THAT THINGY. This lock uses an argument passed to the main function, and subtracts a particular key from it before utilizing it as the file descriptor for a read operation. The trick to this lock is that if the read function is passed a file descriptor of 0, it will read from stdin by default. This will then allow the attacker to input the correct password.

\subsection{Lock 3: Time of Check, Time of Use Lock}
Our third lock was our own design to utilize a classic buffer overflow to overwrite a value before it is used. In this case, we call execve with the values of buffer 2 (which is set to be the same as argv[1]). However, argv[1] was utilized early in the program and needed to be set to a particular value in order to pass lock 2. Not a problem! We have a gets call on a buffer allowing for a buffer overflow. This allows us to overflow into buffer2 and set the value to be ``/bin/shh'' in between when it is used to open lock 2, and when it is used in our third lock to call execve. 
\section{Implementation}
\label{sec:implementation}
The following steps must be followed to get the privilege escalation:
\begin{enumerate}
\item Compile the code with: {\tt gcc -o access\_to\_the\_peanutcave\_reboot access\_to\_the\_peanutcave\_reboot.c.} 

Note that we do not have to disable any default stack protections such as stack canaries or ASLR.
\item The exploit can be run with this command: access\_to\_the\_peanutcave\_reboot 4660.

This includes the first argument as 0x1234, the value it needs to be to clear the second lock.
\item On the first prompt type 4220229742 which will correctly XOR with the random number to provide 0x9000dd09 and unlock the first lock
\item On the second prompt type a padding of size 32 followed by /bin/shh, this will overflow buffer into buffer2 and in doing so allow us to pass /bin/shh as an argument to execve to launch the shell.
\end{enumerate}

\section{Exploitation}
The proof of concept can be run following:

\begin{itemize}
\item If logged as root, change your permission to user with {\tt su - user}
\item You should already be in the {\tt home/user/} folder. If not, execute {\tt cd home/user/}
\item The exploit is the {\tt exploit.sh} file. Simply execute it with {\tt sh exploit.sh}
\item Enjoy your new privileges.
\end{itemize}

\section{Conclusion}
\subsection{Comments on the vulnerability}
It seems that the issue with GNU Screen 4.5 is not a code bug. We did not use code vulnerabilities and techniques we learnt in the CS557 classe to get privilege escalation. The main issue with Screen is a design mistake. The devlopers of the softwares did not conceive that opening the logfiles in root privileges could be used to create root-owned files.

\subsection{Reference}
The proof of concept used in this project is published at {\tt https://www.exploit-db.com/exploits/41154/}.




%\bibliographystyle{ieeetran}
%\bibliography{bib}


\end{document}
